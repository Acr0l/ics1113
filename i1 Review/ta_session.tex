\documentclass{article}

% Packages
\usepackage[utf8]{inputenc}
\usepackage{amsmath}
\usepackage{amssymb}
\usepackage{geometry}
\usepackage{fancyhdr}
\usepackage{lipsum}

% Page setup
\geometry{a4paper, margin=2cm}
\pagestyle{fancy}
\fancyhf{}
\lhead{Class Notes}
\rhead{\today}
\cfoot{\thepage}

% Document content
\begin{document}

\section{Nodos y Arcos}
Considere una red de calles ortogonal de una ciudad en que en cada esquina existe un semáforo. Tenemos
dos amigos: uno de ellos es cartero y el otro revisa la red de semáforos. En el día de hoy cada uno debería
organizar un tour por la ciudad. El primero desea visitar todas las calles de la ciudad caminando de
modo de repartir sus cartas. El semaforero desea visitar todos los semáforos de la red. Modele estos
problemas de modo de obtener rutas de largo mínimo para cada uno


\subsection{Método Vendedor Viajero}
Visitar todos los nodos recorriendo la menor cantidad de arcos posibles.
\subsubsection*{Pasos}
\begin{enumerate}
    \item Modelar el problema
    \item Resolver el problema
    \item Implementar el problema
    \item Testear el problema
\end{enumerate}
\subsubsection*{Variables}
\begin{itemize}
    \item \(x_{ij}\) = 1 si el arco \(i \rightarrow j\) es parte de la solución, 0 en otro caso
\end{itemize}
\subsubsection*{Supuestos}
\begin{itemize}
    \item Tenemos que visitar todos los nodos.
    \item No hay subcircuitos.
\end{itemize}
\subsubsection*{Función Objetivo}
\begin{equation}
    \min \sum_{i=1}^{N} \sum_{j=1}^{N} x_{ij}d_{ij} \quad i\neq j
\end{equation}
\subsubsection*{Restricciones}
\begin{align}
    \sum_{i=1}^{n} x_{ij} &= 1 \quad j = 1, 2, \ldots, N \quad i \neq j \\
    \sum_{j=1}^{n} x_{ij} &= 1 \quad i = 1, 2, \ldots, N \quad j \neq i \\
    \sum_{i \in S} \sum_{j \in S} x_{ij} &\leq \text{Card}(S)-1 \quad i\neq j\quad \forall S \subset V, |S| \leq N-1
\end{align}
\subsubsection*{Naturaleza de las Variables}
\begin{align}
    x_{ij} &\in \{0, 1\} \quad i, j = 1, 2, \ldots, n
\end{align}

\subsection*{Método Cartero Chino}
Visitar todos los arcos minimizando la distancia recorrida.

\subsubsection*{Pasos}
\begin{enumerate}
    \item Modelar el problema
    \item Resolver el problema
\end{enumerate}
\subsubsection*{Conjuntos}
\begin{itemize}
    \item \(N\) = conjunto de nodos
    \item \(d_{ij}\) = distancia entre el noto \((i,j)\in A\)
    \item \((i,j)\in A\) = arco desde el nodo \(i\) al \(j\)
\end{itemize}
\subsubsection*{Variables}
\begin{itemize}
    \item \(x_{ij}\) = cantidad de veces que pasa por el arco \(i\rightarrow j\quad (i,j)\in A\)
\end{itemize}

\subsubsection*{Función Objetivo}
\begin{equation}
    \min \sum_{(i,j)\in A} d_{ij}x_{ij}
\end{equation}
\subsubsection*{Restricciones en Palabras}
\begin{itemize}
    \item De cada nodo sale y entra un arco.
    \item Se quiere pasar por todos los arcos.
\end{itemize}
\subsubsection*{Restricciones}
\begin{align}
    x_{i,j}+x_{j,i} \geq 1 \quad \forall\;(i,j)\in A \\
    \sum_{i} \text{no alcancé a escribir}\\
\end{align}
\section{Convexidad}
Una función par es tal que \(f(x) = f(-x)\). Una función impar es tal que \(f(x) = -f(-x)\).

Para los siguientes problemas, considere una serie de funciones unidimensionales \(f_i(x) = (x-a_i)^i\), 
con \(a_i\in R\) para \(i\in 1,\dots,n\) y una función \(g(x)\) convexa. 
Para cada una de las siguientes funciones, dé un argumento que son convexas o dé un contraejemplo que no lo son.

\begin{itemize}
    \item \(f_1(x)\)
    \item \(\sum_{i=1}^{n} f_i(x)\)
    \item (Asuma que n es par) \(\sum_{i=1}^{n/2} f_{2i}\)
    \item \(fi(x)\cdot g(x)\)
    \item \(\max_{i\in N} f_i(x)\)
\end{itemize}

\subsection*{Solución}
\begin{itemize}
    \item \(f_1(x)=(x-a_i)^1\) es una función lineal, por lo que es convexa.
    \item \(\sum_{i=1}^{n} f_i(x)\) es una función polinómica, por lo que es convexa.
    \item \(\sum_{i=1}^{n/2} f_{2i} \Rightarrow n=3\Rightarrow \sum_{i=1}^3 f_i(x)=(x-a_1)+(x-a_2)^2+(x-a_3)^3\) La función es impar por lo tanto no es convexa (depende de la convexidad del último término).
    \item \(f_i(x)\cdot g(x) \Rightarrow x^2(x-a_1) = x^3-x^2a_1\) Función impar por lo tanto no es convexa
    \item \(\max_{i\in N} f_i(x)\) es una función polinómica, por lo que es convexa.
    \[
    f_{2i+1}(x) = 
    \begin{cases}
        0 & \text{if } x < a_{2i+1} \\
        (x-a_{2i+1})^{2i+1} & \text{if } x \geq a_{2i+1}
    \end{cases}
    \]
    Cada función \(f_i\) es convexa:
    \begin{gather}
        x_1,x_2\in f\\
        x_3 = \lambda x_1 + (1-\lambda)x_2\\
        f_k=\max_{i\in N} f_i(x_3), k\in N\\
        f_k(\lambda x_1+(1-\lambda)x_2) \leq \lambda f_k(x_1) + (1-\lambda)f_k(x_2)
    \end{gather}
\end{itemize}
\end{document}
